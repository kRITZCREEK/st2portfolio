\newglossaryentry{ort}{
  name={Ort},
  description={
    Ein Ort kann ein Raum, ein Standpunkt oder Ausgang sein.
    Dabei besteht ein Ort immer aus Koordinaten UND Bedeutung
  }
}

\newglossaryentry{schritt}{
  name={Schritt},
  description={
  Kleinste logische Einheit die es auf einem Weg gibt. Schritte verbinden
  Abzweigungen, Orte und Ausgänge. Wichtig: Ein Schritt ist definiert durch:
  -- zwei Endpunkte
  -- Länge
  -- Begehbarkeit (Rollstuhlfahrer, alte Menschen, etc.)}
}

\newglossaryentry{weg}{
  name={Weg},
  description={
  Ein Weg ist eine Abfolge von Schritten die gegangen werden müssen um vom
  aktuellen Standpunkt an ein Ziel zu gelangen. Wichtig: Es gibt keine Wege die
  nicht beim aktuellen Standpunkt beginnen und es gibt keine Wege ohne Ziel
  }
}

\newglossaryentry{tuer}{
  name={Tür},
  description={
    Eine Tür trennt zwei Bereiche voneinander. Sie kann entweder manuell oder
    elektronisch gesteuert werden. Weiterhin kann sie geöffnet oder geschlossen
    sein
  }
}

\newglossaryentry{navigation}{
  name={Navigation},
  description={
    Die Navigation zeigt immer den ersten Schritt des aktuellen Weges an.
    Daraus folgt: 
    Der User kennt immer nur den nächsten Schritt
    Der Weg kann jederzeit neu bestimmt werden
    Folgende Navigationsanweisungen sind möglich:
    Rechts/Links/Geradeaus + Längenangabe
    Treppe runter/rauf
    Aufzug betreten + Stockwerk
    }
}

\newglossaryentry{intention}{
  name={Intention},
  description={
    Etwas das der User sucht. Der User sucht keinen Raum sondern das was sich in
    dem Raum befindet
  }
}

\newglossaryentry{mensa}{
  name={Mensa},
  description={
    Ein Ort an dem es Getränke, Kaffee, Snacks und vollständige Mahlzeiten gibt
  }
}

\newglossaryentry{kaffeebar}{
  name={Kaffeebar},
  description={
    Ein Ort an dem es Getränke, Snacks und Kaffee gibt
  }
}

\newglossaryentry{flur}{
  name={Flur},
  description={
    Ein Gang der entweder als Zugang zu einer Tür oder nur als Durchgang genutzt werden kann
  }
}

\newglossaryentry{treppe}{
  name={Treppe},
  description={
    Kann in beide Richtungen genutzt werden um Stockwerk zu wechseln. Für
    Rollstuhlfahrer und Gehbehinderte unbenutzbar. Für alte Menschen sollte ein
    Weg möglichst wenige Treppen enthalten
  }
}

\newglossaryentry{aufzug}{
  name={Aufzug},
  description={
    Ein Gefährt, welches Personen und kleinere Lasten zwischen Stockwerken
    transportieren kann. Besonders wesentlich ist, dass man so Treppen vermeiden kann
  }
}

\newglossaryentry{raum}{
  name={Raum},
  description={
    Alles was in der FH Köln eine Raumnummer hat
  }
}

\newglossaryentry{ausgang}{
  name={Ausgang},
  description={
    Eine Tür die aus einem Gebäude herausführt. Kann entweder in den 
    Innenhof oder aus dem Navigationsbereich führen
  }
}

\newglossaryentry{sammelplatz}{
  name={Sammelplatz},
  description={
    Ein Ort an dem sich die Besucher bei Gefahrsituationen (Brände, Alarm) sammeln
  }
}

\newglossaryentry{information}{
  name={Information},
  description={
    Ein Ort an dem man Informationen und Auskunft über die Räumlichkeiten der
    FH bekommen kann. Außerdem werden hier Transponder zum Öffnen der Türen ausgehändigt
  }
}

\newglossaryentry{labor}{
  name={Labor},
  description={
    Ein Ort an dem Forschung und Wissenschaft praktiziert wird. Ist in der Regel
  abgeschlossen und während der Benutzung beaufsichtigt
  }
}

\newglossaryentry{fhbesucher}{
  name={FH Besucher},
  description={
    Person, welche die Navigations-App verwendet, um sich in der FH zurecht zu finden
  }
}