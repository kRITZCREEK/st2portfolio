\chapter{Resümee}

Im Anschluss an die Kick-Off Veranstaltung, in der ich zu viele,
unstrukturierte Informationen erhielt, hielt ich mit meinem Team eine
erste Lagebesprechung ab.

Wie sich später herausstellte, war unsere anfängliche Annahme, das sich die
genannte Plattformunabhängigkeit auf die zu erstellende App und nicht auf die
Modellierung bezieht, welche durch die Kick-Off Veranstaltung und weiteren
Unterlagen hervorgerufen wurde, falsch. Hier hätten klarere Anforderungen Zeit
sparen und Klarheit schaffen können.

Die Informationsbeschaffung empfand ich als umständlich und fehleranfällig, da
Informationen von vielen unterschiedlichen Stellen (Ilias, Homepages der
Professoren, Vorlesung, Folien, Emails) gesammelt werden mussten. Dies sorgte
für viel Unklarheit und Irrtümer. Eine durchdachtere und einheitliche
Kommunikationspolitik hätte hier Abhilfe schaffen können. Die Frage, wie
ausgereift die App am Ende des Semesters sein sollte, blieb lange offen.
Außerdem fehlten unserem Team oft genauere Rahmenbedingungen und Eingrenzungen.
Schließlich orientierten wir uns von Meilenstein zu Meilenstein an der
Portfolioübersicht.

Die ersten Diagramme (Use-Case-, Activity-Diagram) wurden besonders sorgfältig
angefertigt, mussten dadurch später kaum noch überarbeitet werden und
unterstützten uns immer wieder beim Entwicklungsprozess. Besonders durch das
Use-Case-Diagrammm, konnte man häufig verhindern zu weit von der tatsächlichen
Aufgabe abzuschweifen. Durch Brainstorming konnten Dinge wie Personas, mentale
Modelle und Szenarien, sprich der Nutzungskontext, ohne größere Umstände
erarbeitet werden. Erst im fortgeschrittenen Semester wurde unserem Team
bekannt, dass die Themengebiete Mensch-Computer-Interaktion und Softwaretechnik,
von uns recht strikt getrennt werden konnten, und somit eigentlich nur die
Inhalte der MCI tatsächlich im Programm realisiert werden mussten. Gerade hier
hatte man das Gefühl, dass der Umfang der Aufgabenstellung den Dozenten erst
nachdem die Aufgabe gestellt war, bewusst geworden ist.

Häufig hätte ich mir mehr Zeit gewünscht, was gegebenenfalls durch eine
transparentere und einheitlichere Organisation erreicht hätte werden können.
Während der Meilensteine und besonders durch den Usability-Test viel auf, dass
man als Entwickler schnell den Blick für das Eigentliche Ziel aus den Augen
verliert und das man besonders auf die Unvoreingenommenheit von Dritten
angewiesen ist. Im Team kam es teilweise zu sehr langen Diskussionen über
verschiedene Designentscheidungen der App, welche durch die Befragen oder Tests
schneller hätten geklärt werden können. Für mich stellte sich heraus das die
Meinung der Stakeholder regelmäßig eingeholt werden muss um die Anforderungen an
ein Projekt angemessen erfüllen zu können.

Unser Tooling (Astah-Community) für die Erstellung der UML-Diagramm war
durchwegs brauchbar und erleichterte die Arbeit mit UML.